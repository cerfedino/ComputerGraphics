\documentclass[tikz,14pt,fleqn]{article}


\usepackage[utf8]{inputenc}
\usepackage[margin=1in]{geometry}
\usepackage[titletoc,title]{appendix}
\usepackage{latexsym}
\usepackage{amssymb}
\usepackage{gensymb}
\usepackage{amsmath}
\usepackage{amsfonts}
\usepackage{multicol}
\usepackage{graphicx}
\usepackage{fancyhdr}
\usepackage[linguistics]{forest}
\usepackage{colortbl}
\usepackage[dvipsnames]{xcolor}
\usepackage{pdfpages}
\usepackage{wrapfig}

%% For plotting
\usepackage{pgfplots}
\pgfplotsset{width=10cm,compat=1.9}
\usepgfplotslibrary{external}
\tikzexternalize
%%
\usepackage{dirtree}
\usepackage{subcaption}
\usepackage{xifthen}% provides \isempty test
\usepackage{glossaries}

\captionsetup[subfigure]{labelformat=empty}
\definecolor{color1}{HTML}{0B0C10}
\definecolor{color2}{HTML}{1F2833}
\definecolor{color3}{HTML}{C5C6C7}
\definecolor{color4}{HTML}{66FCF1}
\definecolor{color5}{HTML}{45A29E}

\pagestyle{fancy}
\fancyhf{}
%%%%%%%%%%%%%%%%%%%%%%%%%%%%
%% VARIABLES
\newcommand\namesurname{Albert Cerfeda\\Alessandro Gobbetti}
\newcommand\assignment{Assignment 1}

\newcommand\subject{Computer Graphics}
\newcommand\documentdate{23.09.2022}

% Title content
%%%%%%%%%%%%%%%%%%%%%%%%%%%%
\rhead{\assignment}
\lhead{\namesurname}
%%%%%%%%%%%%%%%%%%%%%%%%%%%%
\rfoot{Page \thepage}
\setlength{\parindent}{0pt}

\newcommand\xdownarrow[1][2ex]{%
   \mathrel{\rotatebox{90}{$\xleftarrow{\rule{#1}{0pt}}$}}
}

\begin{document}

\begin{titlepage}
   \begin{center}
       \vspace*{1cm}

       \textbf{\Large{Homework Assignment}}

       \vspace{0.5cm}
        \textbf{\subject}\\[5mm]
       \assignment
        
            
       \vspace{1.8cm}

        \namesurname
       \tableofcontents

       \vspace*{\fill}
     
       \includegraphics[width=0.4\textwidth]{logo.png}
       
        \documentdate \\
        Università della Svizzera italiana\\
        Faculty of Informatics\\
        Switzerland\\

   \end{center}
\end{titlepage}


%%%%%%%%%%%%%%%%%%%%%%%%%%%%%%%%%%%%%%%%%%%%%%%%%%%%
%%%%%%  CONTENT START   %%%%%%%%%%%%%%%%%%%%%%%%%%%%
% color	white, black, red, green, blue, cyan, magenta, yellow
% thickness	ultra thin, very thin, thin, thick, very thick, ultra thick
% dots (/loosely/densely) + dotted, dashed, dashdotted
\section{Exercise 1}
\[
x = (\sqrt{2}, 1, 0)^T
\]
\[ y = (1,1,1)^T\]
\[
A =
\begin{pmatrix}
1 & 1 & 1 \\
2 & 2 & 1 \\
-1 & -3 & -3 \\
\end{pmatrix}
\]
\subsubsection*{Task 1}
\[
\langle x,y \rangle = \cos\alpha\| x\|\| y \|
\]
\[
\Rightarrow \cos\alpha = \frac{\langle x,y \rangle}{\| x\|\| y \|}
\]\\
\[
\langle x,y \rangle = \sqrt{2}+1
\]
\begin{align*}
    \cos\alpha &= \frac{\sqrt{2}+1}{\sqrt{3}*\sqrt{3}}\\
    &= \frac{\sqrt{2} +1}{3}
\end{align*}
\[
\]

% ----------------------------------------------------------
\subsubsection*{Task 2}
\[ \hat{z} = \frac{x \times y}{\|z\|} \]
\[
z_1 =   
\begin{bmatrix}
x_2 & y_2\\
x_3 & y_3
\end{bmatrix} =
\begin{bmatrix}
1 & 1\\
0 & 1
\end{bmatrix} = 1
\]
\[
z_2 = -
\begin{bmatrix}
x_1 & y_1\\
x_3 & y_3
\end{bmatrix} = -
\begin{bmatrix}
\sqrt{2} & 1\\
0 & 1
\end{bmatrix} = -\sqrt{2}
\]
\[
z_1 = 
\begin{bmatrix}
x_1 & y_1\\
x_2 & y_2
\end{bmatrix} =
\begin{bmatrix}
\sqrt{2} & 1\\
1 & 1
\end{bmatrix} = \sqrt{2} - 1
\]

Thus:
\begin{align*}
\hat{z} &= \frac{\begin{pmatrix}1 & -\sqrt{2} & \sqrt{2}-1\end{pmatrix}^T}{\sqrt{\langle z,z \rangle}} \\
        &= \frac{\begin{pmatrix}1 & -\sqrt{2} & \sqrt{2}-1\end{pmatrix}^T}{\sqrt{3+2+1-2\sqrt{2}}} \\
        &= \frac{\begin{pmatrix}1 & -\sqrt{2} & \sqrt{2}-1\end{pmatrix}^T}{\sqrt{6-2\sqrt{2}}} 
\end{align*}
% ----------------------------------------------------------
\subsubsection*{Task 3}
\begin{align*}
    u &= A z \\
    &= \begin{pmatrix}
1 & 1 & 1 \\
2 & 2 & 1 \\
-1 & -3 & -3 \\
\end{pmatrix} \cdot \frac{\begin{pmatrix}1 & -\sqrt{2} & \sqrt{2}-1\end{pmatrix}^T}{\sqrt{6-2\sqrt{2}}} 
\end{align*}


\begin{align*}
u &= \frac{1}{\sqrt{6-2\sqrt{2}}}\begin{pmatrix}
1 & 1 & 1 \\
2 & 2 & 1 \\
-1 & -3 & -3 \\
\end{pmatrix} 
\cdot
\begin{pmatrix}
1 \\
-\sqrt{2} \\
\sqrt{2}-1 \\
\end{pmatrix} \\
&=
\frac{1}{\sqrt{6-2\sqrt{2}}} 
\begin{pmatrix}
1 \cdot 1 + 1 \cdot (-\sqrt{2}) + 1 \cdot (\sqrt{2}-1) \\
2 \cdot 1 + 2 \cdot (-\sqrt{2}) + 1 \cdot (\sqrt{2}-1) \\
-1 \cdot 1 -3 \cdot (-\sqrt{2}) -3 \cdot (\sqrt{2}-1) \\
\end{pmatrix} \\
&=
\frac{1}{\sqrt{6-2\sqrt{2}}}
\begin{pmatrix}
1 -\sqrt{2} + \sqrt{2}-1 \\
2 -2\sqrt{2} + \sqrt{2}-1 \\
-1 +3\sqrt{2} -3\sqrt{2} +3) \\
\end{pmatrix} \\
&=
\frac{1}{\sqrt{6-2\sqrt{2}}}
\begin{pmatrix}
0 \\
1 - \sqrt{2} \\
2 \\
\end{pmatrix}
\end{align*}


\section{Exercise 2}
\begin{align*}
    c = \begin{pmatrix}
    1 & 1 & 1
    \end{pmatrix}^T
\end{align*}
\begin{align*}
    \|l\| = \sqrt{\langle l,l \rangle} = \sqrt{3}
\end{align*}
\begin{align*}
    \alpha &= 180 - 90 -\arccos{\frac{r}{\|l\|}} \\ 
    &= 90 -\arccos{\frac{\frac{\sqrt{2}}{2}}{\sqrt{3}}} \\
    &= 90 -\arccos{\frac{\sqrt{2}}{2\sqrt{3}}} \\
    &= 90 - 65.91\degree = 24.09\degree
\end{align*}
% $$\underline{x} = \binom{0}{3},
% \underline{y} = \binom{-3}{0}$$
% Let us plot the vectors:
% \begin{figure}[h!]
% \begin{tikzpicture}
%     \node[label=x] at (1.2,-0.2){};
%     \draw[->] (-1,0) -- (1.5,0);
%     \node[label=y] at (0.2,2.5){};
%     \draw[->] (0,-1) -- (0,3);
%     \node[label=$0$] at (0.2,-0.6){};
%     \filldraw [black] (0,0) circle (1.2pt);
    
%     \node[black] at (-0.2,2) { \underline{\textbf{x}} };
%     \draw[->, black, very thick] (0 , 0) -- (0 , 3);
    
%     \node[black] at (-2,0.2) { \underline{\textbf{y}} };
%     \draw[->, black, very thick] (0 , 0) -- (-3,0);
% \end{tikzpicture}
% \end{figure}

\clearpage


\end{document}
